%\usepackage{titlesec}
% PACKAGE INCLUSION
\usepackage{amsbsy,amssymb,amsmath,amsfonts} % Math and related rendering
\allowdisplaybreaks
\usepackage{tikz-cd,tikz-3dplot,circuitikz} % For graphical drawing
%\usepackage{natbib} % For citation, using numbering style. 
\usepackage{graphicx}
\usepackage{geometry} % configure the geometry page
\usepackage{bbm} % Better bold (for example, \mathbb{1} does not work for font not alphabetical)
\usepackage{bm} % Bold symbol 
\usepackage{tikz} % for normal graphing
\usepackage{enumitem} % for customizing options for lists, enumerations, and descriptions
\usepackage{fancyhdr} % configuring header
\usepackage{parskip} % For paragraph spacing
\usepackage{caption} % captioning options
\usepackage{mathtools} % using this to fix \multlined environment altogether
\usepackage{parnotes}% take this one for marginnote substitute (or the reverse because this one looks so much better)
\usepackage{tabularx} % better tabular
\usepackage{braket} % For bra-ket notation
%\usepackage{fontspec} % does fontspec works for memoir? 
\usepackage{float} % more float control
\usepackage[dvipsnames,svgnames,x11names]{xcolor} % color schema, definitely should have included this
\usetikzlibrary{matrix}
\usepackage{thmtools} % hooks and stuff for listoftheorem
%\DisemulatePackage{showidx} %(2)
%\usepackage{makeidx} % I HATE THIS
%\usepackage{showidx} % I HATE THIS
%\usepackage{subfigure}
\usepackage{caption} % uh, caption
\usepackage{nicematrix} % for drawing matrix
\usepackage{subcaption} % This and the two right above it is required for multi-figure display, so yeah.
%\setcounter{tocdepth}{2}
\usepackage{stmaryrd} % double brackets for integer intervals]
\usepackage{algorithm2e}
\usepackage{algorithmicx}
\usepackage{array}
\usepackage{booktabs}
\newcolumntype{P}[1]{>{\raggedright\arraybackslash}p{#1}}
%-------------------------------%
% CONFIGURING NATBIB
%\usepackage{natbib} 
%-------------------------------%
% NEXT IS INDEX CONFIG
%%%%%%%%%%%%%%%%%%%%%%%%%%%
\makeatletter
\renewcommand{\index}[1]{%
  \oldindex{#1}%
  \if@reversemargin
    \marginpar{\raggedleft\small#1}%
  \else
    \marginpar{\raggedright\small#1}%
  \fi
}
\makeatother

%----------------------------------------------------------------------------------% 
% SMALL SETTING
\usepackage{pgfplots} % for something else
\usetikzlibrary{patterns}       % For custom patterns
\usetikzlibrary{shadings}       % For gradient fills
\usetikzlibrary{shapes.geometric} % For geometric shapes
\usetikzlibrary{calc}           % For coordinate calculations
\usetikzlibrary{positioning}    % For node positioning
\usetikzlibrary{decorations.pathreplacing} % For path decorations
\usetikzlibrary{fit} % For fitting shapes around nodes
\newcommand{\calx}[1]{\mathcal{#1}}
\newcommand{\mynote}[1]{\medskip\par\textbf{\small Note}\quad\setlength{\extrarowheight}{2pt}\begin{tabularx}{\linegoal}{X}
\Xhline{1pt}
\rowcolor{WhiteSmoke!80!Lavender}#1 \\
\Xhline{1pt}
\end{tabularx}}

\makeatletter
\pgfutil@ifundefined{pgf@pattern@name@_xg1qse1zm}{
  \pgfdeclarepatternformonly[\mcThickness,\mcSize]{_xg1qse1zm}
  {\pgfqpoint{0pt}{-\mcThickness}}
  {\pgfpoint{\mcSize}{\mcSize}}
  {\pgfpoint{\mcSize}{\mcSize}}
  {
    \pgfsetcolor{\tikz@pattern@color}
    \pgfsetlinewidth{\mcThickness}
    \pgfpathmoveto{\pgfqpoint{0pt}{\mcSize}}
    \pgfpathlineto{\pgfpoint{\mcSize+\mcThickness}{-\mcThickness}}
    \pgfusepath{stroke}
  }
}
\makeatother

\makeatletter
\pgfutil@ifundefined{pgf@pattern@name@_lbrsyyeax}{
  \pgfdeclarepatternformonly[\mcThickness,\mcSize]{_lbrsyyeax}
  {\pgfqpoint{0pt}{0pt}}
  {\pgfpoint{\mcSize+\mcThickness}{\mcSize+\mcThickness}}
  {\pgfpoint{\mcSize}{\mcSize}}
  {
    \pgfsetcolor{\tikz@pattern@color}
    \pgfsetlinewidth{\mcThickness}
    \pgfpathmoveto{\pgfqpoint{0pt}{0pt}}
    \pgfpathlineto{\pgfpoint{\mcSize+\mcThickness}{\mcSize+\mcThickness}}
    \pgfusepath{stroke}
  }
}
\makeatother
%%%%%%%%%%%%%%%%%%%%%%%%%%%
\usepackage{hyperref} % For referencing
%\hypersetup{
%  colorlinks=true,
%  allcolors=blue
%}

\newcommand{\Ap}{A_{\sim p}}
\newcommand{\Aq}{A_{\sim q}}
\newcommand{\Xp}{X_{\sim p}}
\newcommand{\Xq}{X_{\sim q}}

\newcommand{\xp}{x_{\sim p}}
\newcommand{\xq}{x_{\sim q}}
\newcommand{\yp}{y_{\sim p}}
\newcommand{\yq}{y_{\sim q}}
\renewcommand{\wp}{w_{\sim p}}
\newcommand{\wq}{w_{\sim q}}

%%%%%%%%%%%%%%%%%%%%%%%%%%%%%%%%%%%%%%%%%%%%%%%%%%%%%%% COLOR BOX CONFIG
%\theoremstyle{plain}
%\newtheorem{definition}{Definition}[section]
%\newtheorem{theorem}{Theorem}[section]
\newtheorem{col}{Corollary}[subsection]
%\newtheorem{conjecture}{Conjecture}[section]
\newtheorem{setting}{Setting}[section]
%\newtheorem{proposition}[theorem]{Proposition}
%\newtheorem{lemma}[theorem]{Lemma}
\newtheorem{assumption}[theorem]{Assumption}
\newtheorem{assume}{Assumption}[subsection]
%\newtheorem{remark}[theorem]{Remark}
\newtheorem{hypothesis}{Hypothesis}[section]
%\newtheorem{axiom}{Axiom}[section]
\newtheorem{question}{Question}[section]
%\newtheorem{example}{Example}[section]
\newtheorem{note}{Note}[section]
%%%%%%%%%%%%%%%%%%%%%%%%%%%%

% nicer spacing
\renewcommand{\arraystretch}{1.35} % vertical row padding
\setlength{\tabcolsep}{8pt}        % horizontal cell padding

% helper column types
\newcolumntype{L}[1]{>{\raggedright\arraybackslash}p{#1}}
\newcolumntype{Y}{>{\raggedright\arraybackslash}X}