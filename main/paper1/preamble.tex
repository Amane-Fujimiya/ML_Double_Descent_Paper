%\usepackage{titlesec}
% PACKAGE INCLUSION
\usepackage{titlesec}
\usepackage{amsbsy,amssymb,amsmath,amsfonts,amsthm} % Math and related rendering
\allowdisplaybreaks
\usepackage{tikz-cd,tikz-3dplot,circuitikz} % For graphical drawing
%\usepackage{natbib} % For citation, using numbering style. 
\usepackage{graphicx}
%\usepackage{geometry} % configure the geometry page
\usepackage{bbm} % Better bold (for example, \mathbb{1} does not work for font not alphabetical)
\usepackage{bm} % Bold symbol 
\usepackage{tikz} % for normal graphing
\usepackage{enumitem} % for customizing options for lists, enumerations, and descriptions
\usepackage{fancyhdr} % configuring header
\usepackage{parskip} % For paragraph spacing
\usepackage{caption} % captioning options
\usepackage{mathtools} % using this to fix \multlined environment altogether
\usepackage{parnotes}% take this one for marginnote substitute (or the reverse because this one looks so much better)
\usepackage{tabularx} % better tabular
\usepackage{braket} % For bra-ket notation
%\usepackage{fontspec} % does fontspec works for memoir? 
\usepackage{float} % more float control
\usepackage[dvipsnames,svgnames,x11names]{xcolor} % color schema, definitely should have included this
\usetikzlibrary{matrix}
\usepackage{thmtools} % hooks and stuff for listoftheorem
\usepackage[utf8]{inputenc} % allow utf-8 input
\usepackage{url}            % simple URL typesetting
\usepackage{nicefrac}       % compact symbols for 1/2, etc.
\usepackage{microtype}      % microtypography
\usepackage{stmaryrd}
\usepackage{csquotes}
%\DisemulatePackage{showidx} %(2)
%\usepackage{makeidx} % I HATE THIS
%\usepackage{showidx} % I HATE THIS
%\usepackage{subfigure}
\usepackage{caption} % uh, caption
\usepackage{nicematrix} % for drawing matrix
\usepackage{subcaption} % This and the two right above it is required for multi-figure display, so yeah.
%\setcounter{tocdepth}{2}
%\usepackage{stmaryrd} % double brackets for integer intervals]
\usepackage{algorithm2e}
\usepackage{algorithmicx}
\usepackage{array}
\usepackage{booktabs}
\newcolumntype{P}[1]{>{\raggedright\arraybackslash}p{#1}}
%-------------------------------%
% CONFIGURING NATBIB
\usepackage{natbib} 
%-------------------------------%
%%%%%%%%%%%%%%%%%%%%%%%%%%%
% SECTION -- explicit full form:
% \titleformat{<command>}[<shape>]{<format>}{<label>}{<sep>}{<before-code>}[<after-code>]                                 % code after title text (optional)

% SUBSECTION
%\titleformat{\subsection}[hang]
%  {\sffamily\normalsize}
%  {\thesubsection}
%  {1em}
%  {}
%  []

% SUBSUBSECTION
%\titleformat{\subsubsection}[hang]
%  {\sffamily\normalsize}
%  {\thesubsubsection}
%  {1em}
%  {}
%  []
%%%%%%%%%%%%%%%%%%%%%%%%%%%
%----------------------------------------------------------------------------------% 
% SMALL SETTING
\usepackage{pgfplots} % for something else
\usetikzlibrary{patterns}       % For custom patterns
\usetikzlibrary{shadings}       % For gradient fills
\usetikzlibrary{shapes.geometric} % For geometric shapes
\usetikzlibrary{calc}           % For coordinate calculations
\usetikzlibrary{positioning}    % For node positioning
\usetikzlibrary{decorations.pathreplacing} % For path decorations
\usetikzlibrary{fit} % For fitting shapes around nodes
\newcommand{\calx}[1]{\mathcal{#1}}
\newcommand{\mynote}[1]{\medskip\par\textbf{\small Note}\quad\setlength{\extrarowheight}{2pt}\begin{tabularx}{\linegoal}{X}
\Xhline{1pt}
\rowcolor{WhiteSmoke!80!Lavender}#1 \\
\Xhline{1pt}
\end{tabularx}}

\makeatletter
\pgfutil@ifundefined{pgf@pattern@name@_xg1qse1zm}{
  \pgfdeclarepatternformonly[\mcThickness,\mcSize]{_xg1qse1zm}
  {\pgfqpoint{0pt}{-\mcThickness}}
  {\pgfpoint{\mcSize}{\mcSize}}
  {\pgfpoint{\mcSize}{\mcSize}}
  {
    \pgfsetcolor{\tikz@pattern@color}
    \pgfsetlinewidth{\mcThickness}
    \pgfpathmoveto{\pgfqpoint{0pt}{\mcSize}}
    \pgfpathlineto{\pgfpoint{\mcSize+\mcThickness}{-\mcThickness}}
    \pgfusepath{stroke}
  }
}
\makeatother

\makeatletter
\pgfutil@ifundefined{pgf@pattern@name@_lbrsyyeax}{
  \pgfdeclarepatternformonly[\mcThickness,\mcSize]{_lbrsyyeax}
  {\pgfqpoint{0pt}{0pt}}
  {\pgfpoint{\mcSize+\mcThickness}{\mcSize+\mcThickness}}
  {\pgfpoint{\mcSize}{\mcSize}}
  {
    \pgfsetcolor{\tikz@pattern@color}
    \pgfsetlinewidth{\mcThickness}
    \pgfpathmoveto{\pgfqpoint{0pt}{0pt}}
    \pgfpathlineto{\pgfpoint{\mcSize+\mcThickness}{\mcSize+\mcThickness}}
    \pgfusepath{stroke}
  }
}
\makeatother
%%%%%%%%%%%%%%%%%%%%%%%%%%%
\usepackage{hyperref} % For referencing
%\hypersetup{
%  colorlinks=true,
%  allcolors=blue
%}

\newcommand{\Ap}{A_{\sim p}}
\newcommand{\Aq}{A_{\sim q}}
\newcommand{\Xp}{X_{\sim p}}
\newcommand{\Xq}{X_{\sim q}}

\newcommand{\xp}{x_{\sim p}}
\newcommand{\xq}{x_{\sim q}}
\newcommand{\yp}{y_{\sim p}}
\newcommand{\yq}{y_{\sim q}}
\renewcommand{\wp}{w_{\sim p}}
\newcommand{\wq}{w_{\sim q}}

%%%%%%%%%%%%%%%%%%%%%%%%%%%%%%%%%%%%%%%%%%%%%%%%%%%%%%% COLOR BOX CONFIG
%\theoremstyle{plain}
\newtheorem{definition}{Definition}[section]
\newtheorem{theorem}{Theorem}[section]
\newtheorem{col}{Corollary}[subsection]
\newtheorem{conjecture}{Conjecture}[section]
\newtheorem{setting}{Setting}[section]
\newtheorem{proposition}[theorem]{Proposition}
%\newtheorem{lemma}[theorem]{Lemma}
\newtheorem{assumption}[theorem]{Assumption}
\newtheorem{assume}{Assumption}[subsection]
%\newtheorem{remark}[theorem]{Remark}
\newtheorem{hypothesis}{Hypothesis}[section]
%\newtheorem{axiom}{Axiom}[section]
\newtheorem{question}{Question}[section]
%\newtheorem{example}{Example}[section]
\newtheorem{note}{Note}[section]
%%%%%%%%%%%%%%%%%%%%%%%%%%%%

%% nicer spacing
%\renewcommand{\arraystretch}{1.35} % vertical row padding
%\setlength{\tabcolsep}{8pt}        % horizontal cell padding

% helper column types
\newcolumntype{L}[1]{>{\raggedright\arraybackslash}p{#1}}
\newcolumntype{Y}{>{\raggedright\arraybackslash}X}
\newcommand{\dataset}{{\cal D}}
\newcommand{\fracpartial}[2]{\frac{\partial #1}{\partial  #2}}
%########################################
%# Tables
\usepackage{threeparttable}
\usepackage{ragged2e}
\usepackage[leftcaption]{sidecap}
\usepackage[breakable,theorems,skins]{tcolorbox}
%################################################################################
%\tcbuselibrary{breakable} % Breakable criteria library for tcolorbox - enable multipage breaks
%########################################
%# Default setting and global option
\tcbset{%
  breakable,
  enhanced jigsaw,
  parbox=false,
  nobeforeafter,
  before skip=10pt,
  after skip=10pt,
  arc=2pt,
}
%########################################
%# Box definition (match environment)
\tcolorboxenvironment{theorem}{
  breakable,
  colback=blue!5!white,
  boxrule=0pt,
  boxsep=1pt,
  left=2pt,right=2pt,top=2pt,bottom=2pt,
  oversize=2pt,
  enhanced jigsaw,
}
\tcolorboxenvironment{definition}{
  colback=gray!10!white,
  boxrule=0pt,
  boxsep=1pt,
  left=2pt, right=2pt, top=2pt, bottom=2pt,
  oversize=2pt,
  %sharp corners,
}
\tcolorboxenvironment{col}{
  colback=orange!10!white,
  boxrule=0pt,
  boxsep=1pt,
  left=2pt, right=2pt, top=2pt, bottom=2pt,
  oversize=2pt,
  %sharp corners,
}

\tcolorboxenvironment{conjecture}{
  colback=red!10!white,
  boxrule=0pt,
  boxsep=1pt,
  left=2pt, right=2pt, top=2pt, bottom=2pt,
  oversize=2pt,
  %sharp corners,
}

\tcolorboxenvironment{setting}{
  colback=blue!40!orange!20!white, % Mix blue (40%) and orange (20%)
  boxrule=0pt,
  boxsep=1pt,
  left=2pt, right=2pt, top=2pt, bottom=2pt,
  oversize=2pt,
  %sharp corners,
}
% Proposition (green)
\tcolorboxenvironment{proposition}{
  colback=green!5!white,
  boxrule=0pt,
  boxsep=1pt,
  left=2pt, right=2pt, top=2pt, bottom=2pt,
  oversize=2pt,
  %sharp corners,
}

% Lemma (orange -- toned down)
\tcolorboxenvironment{lemma}{
  colback=orange!10!white,
  boxrule=0pt,
  boxsep=1pt,
  left=2pt, right=2pt, top=2pt, bottom=2pt,
  oversize=2pt,
  %sharp corners,
}

% Assumption (purple)
\tcolorboxenvironment{assumption}{
  colback=purple!5!white,
  boxrule=0pt,
  boxsep=1pt,
  left=2pt, right=2pt, top=2pt, bottom=2pt,
  oversize=2pt,
  %sharp corners,
}

% Assume (a variant of Assumption with a slightly different shade)
\tcolorboxenvironment{assume}{
  colback=purple!10!white,
  boxrule=0pt,
  boxsep=1pt,
  left=2pt, right=2pt, top=2pt, bottom=2pt,
  oversize=2pt,
  %sharp corners,
}

% Remark (gray, very neutral)
\tcolorboxenvironment{remark}{
  colback=gray!5!white,
  boxrule=0pt,
  boxsep=1pt,
  left=2pt, right=2pt, top=2pt, bottom=2pt,
  oversize=2pt,
  %sharp corners,
}

% Hypothesis (cyan)
\tcolorboxenvironment{hypothesis}{
  colback=cyan!10!white,
  boxrule=0pt,
  boxsep=1pt,
  left=2pt, right=2pt, top=2pt, bottom=2pt,
  oversize=2pt,
  %sharp corners,
}

% Axiom (yellow -- with extra white to reduce brightness)
\tcolorboxenvironment{axiom}{
  colback=yellow!10!white,
  boxrule=0pt,
  boxsep=1pt,
  left=2pt, right=2pt, top=2pt, bottom=2pt,
  oversize=2pt,
  %sharp corners,
}

% Question (magenta -- toned down)
\tcolorboxenvironment{question}{
  colback=magenta!10!white,
  boxrule=0pt,
  boxsep=1pt,
  left=2pt, right=2pt, top=2pt, bottom=2pt,
  oversize=2pt,
  %sharp corners,
}

% Interject (red -- toned down)
\tcolorboxenvironment{interject}{
  colback=red!10!white,
  boxrule=0pt,
  boxsep=1pt,
  left=2pt, right=2pt, top=2pt, bottom=2pt,
  oversize=2pt,
  %sharp corners,
}

% Example (teal)
\tcolorboxenvironment{note}{
  colback=teal!10!white,
  boxrule=0pt,
  boxsep=1pt,
  left=2pt, right=2pt, top=2pt, bottom=2pt,
  oversize=2pt,
  %sharp corners,
  enhanced jigsaw,
}
%################################################################################
