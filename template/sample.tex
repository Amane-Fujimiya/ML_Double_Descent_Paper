\documentclass[preview, border=5pt,10pt]{standalone}
\usepackage[preprint]{jmlr2e}
\usepackage{blindtext}
\usepackage{subfig}
% Any additional packages needed should be included after jmlr2e.
% Note that jmlr2e.sty includes epsfig, amssymb, natbib and graphicx,
% and defines many common macros, such as 'proof' and 'example'.
%
% It also sets the bibliographystyle to plainnat; for more information on
% natbib citation styles, see the natbib documentation, a copy of which
% is archived at http://www.jmlr.org/format/natbib.pdf

% Available options for package jmlr2e are:
%
%   - abbrvbib : use abbrvnat for the bibliography style
%   - nohyperref : do not load the hyperref package
%   - preprint : remove JMLR specific information from the template,
%         useful for example for posting to preprint servers.
%
% Example of using the package with custom options:
%
% \usepackage[abbrvbib, preprint]{jmlr2e}

\usepackage[utf8]{inputenc} % allow utf-8 input
\usepackage[T1]{fontenc}    % use 8-bit T1 fonts
\usepackage{url}            % simple URL typesetting
\usepackage{booktabs}       % professional-quality tables
\usepackage{amsfonts}       % blackboard math symbols
\usepackage{nicefrac}       % compact symbols for 1/2, etc.
\usepackage{microtype}      % microtypography

\usepackage{titlesec}
% PACKAGE INCLUSION
\usepackage{amsbsy,amssymb,amsmath,amsfonts,amsthm} % Math and related rendering
\allowdisplaybreaks
\usepackage{tikz-cd,tikz-3dplot,circuitikz} % For graphical drawing
%\usepackage{natbib} % For citation, using numbering style. 
\usepackage{graphicx}
\usepackage{geometry} % configure the geometry page
\usepackage{bbm} % Better bold (for example, \mathbb{1} does not work for font not alphabetical)
\usepackage{bm} % Bold symbol 
\usepackage{tikz} % for normal graphing
\usepackage{enumitem} % for customizing options for lists, enumerations, and descriptions
\usepackage{fancyhdr} % configuring header
\usepackage{parskip} % For paragraph spacing
\usepackage{caption} % captioning options
\usepackage{mathtools} % using this to fix \multlined environment altogether
\usepackage{parnotes}% take this one for marginnote substitute (or the reverse because this one looks so much better)
\usepackage{tabularx} % better tabular
\usepackage{braket} % For bra-ket notation
%\usepackage{fontspec} % does fontspec works for memoir? 
\usepackage{float} % more float control
\usepackage[dvipsnames,svgnames,x11names]{xcolor} % color schema, definitely should have included this
\usetikzlibrary{matrix}
\usepackage{thmtools} % hooks and stuff for listoftheorem
%\DisemulatePackage{showidx} %(2)
%\usepackage{makeidx} % I HATE THIS
%\usepackage{showidx} % I HATE THIS
%\usepackage{subfigure}
\usepackage{caption} % uh, caption
\usepackage{nicematrix} % for drawing matrix
\usepackage{subcaption} % This and the two right above it is required for multi-figure display, so yeah.
%\setcounter{tocdepth}{2}
\usepackage{stmaryrd} % double brackets for integer intervals]
\usepackage{algorithm2e}
\usepackage{algorithmicx}
\usepackage{array}
\usepackage{booktabs}
\newcolumntype{P}[1]{>{\raggedright\arraybackslash}p{#1}}
%-------------------------------%
% CONFIGURING NATBIB
\usepackage{natbib} 
%-------------------------------%
% NEXT IS INDEX CONFIG
%%%%%%%%%%%%%%%%%%%%%%%%%%%
\makeatletter
\renewcommand{\index}[1]{%
  \oldindex{#1}%
  \if@reversemargin
    \marginpar{\raggedleft\small#1}%
  \else
    \marginpar{\raggedright\small#1}%
  \fi
}
\makeatother

%----------------------------------------------------------------------------------% 
% SMALL SETTING
\usepackage{pgfplots} % for something else
\usetikzlibrary{patterns}       % For custom patterns
\usetikzlibrary{shadings}       % For gradient fills
\usetikzlibrary{shapes.geometric} % For geometric shapes
\usetikzlibrary{calc}           % For coordinate calculations
\usetikzlibrary{positioning}    % For node positioning
\usetikzlibrary{decorations.pathreplacing} % For path decorations
\usetikzlibrary{fit} % For fitting shapes around nodes
\newcommand{\calx}[1]{\mathcal{#1}}
\newcommand{\mynote}[1]{\medskip\par\textbf{\small Note}\quad\setlength{\extrarowheight}{2pt}\begin{tabularx}{\linegoal}{X}
\Xhline{1pt}
\rowcolor{WhiteSmoke!80!Lavender}#1 \\
\Xhline{1pt}
\end{tabularx}}

\makeatletter
\pgfutil@ifundefined{pgf@pattern@name@_xg1qse1zm}{
  \pgfdeclarepatternformonly[\mcThickness,\mcSize]{_xg1qse1zm}
  {\pgfqpoint{0pt}{-\mcThickness}}
  {\pgfpoint{\mcSize}{\mcSize}}
  {\pgfpoint{\mcSize}{\mcSize}}
  {
    \pgfsetcolor{\tikz@pattern@color}
    \pgfsetlinewidth{\mcThickness}
    \pgfpathmoveto{\pgfqpoint{0pt}{\mcSize}}
    \pgfpathlineto{\pgfpoint{\mcSize+\mcThickness}{-\mcThickness}}
    \pgfusepath{stroke}
  }
}
\makeatother

\makeatletter
\pgfutil@ifundefined{pgf@pattern@name@_lbrsyyeax}{
  \pgfdeclarepatternformonly[\mcThickness,\mcSize]{_lbrsyyeax}
  {\pgfqpoint{0pt}{0pt}}
  {\pgfpoint{\mcSize+\mcThickness}{\mcSize+\mcThickness}}
  {\pgfpoint{\mcSize}{\mcSize}}
  {
    \pgfsetcolor{\tikz@pattern@color}
    \pgfsetlinewidth{\mcThickness}
    \pgfpathmoveto{\pgfqpoint{0pt}{0pt}}
    \pgfpathlineto{\pgfpoint{\mcSize+\mcThickness}{\mcSize+\mcThickness}}
    \pgfusepath{stroke}
  }
}
\makeatother
%%%%%%%%%%%%%%%%%%%%%%%%%%%
\usepackage{hyperref} % For referencing
%\hypersetup{
%  colorlinks=true,
%  allcolors=blue
%}

\newcommand{\Ap}{A_{\sim p}}
\newcommand{\Aq}{A_{\sim q}}
\newcommand{\Xp}{X_{\sim p}}
\newcommand{\Xq}{X_{\sim q}}

\newcommand{\xp}{x_{\sim p}}
\newcommand{\xq}{x_{\sim q}}
\newcommand{\yp}{y_{\sim p}}
\newcommand{\yq}{y_{\sim q}}
\renewcommand{\wp}{w_{\sim p}}
\newcommand{\wq}{w_{\sim q}}

%%%%%%%%%%%%%%%%%%%%%%%%%%%%%%%%%%%%%%%%%%%%%%%%%%%%%%% COLOR BOX CONFIG
\theoremstyle{plain}
\newtheorem{definition}{Definition}[section]
\newtheorem{theorem}{Theorem}[section]
\newtheorem{col}{Corollary}[subsection]
\newtheorem{conjecture}{Conjecture}[section]
\newtheorem{setting}{Setting}[section]
\newtheorem{proposition}[theorem]{Proposition}
\newtheorem{lemma}[theorem]{Lemma}
\newtheorem{assumption}[theorem]{Assumption}
\newtheorem{assume}{Assumption}[subsection]
\newtheorem{remark}[theorem]{Remark}
\newtheorem{hypothesis}{Hypothesis}[section]
\newtheorem{axiom}{Axiom}[section]
\newtheorem{question}{Question}[section]
\newtheorem{example}{Example}[section]
\newtheorem{note}{Note}[section]
%%%%%%%%%%%%%%%%%%%%%%%%%%%%

% nicer spacing
\renewcommand{\arraystretch}{1.35} % vertical row padding
\setlength{\tabcolsep}{8pt}        % horizontal cell padding

% helper column types
\newcolumntype{L}[1]{>{\raggedright\arraybackslash}p{#1}}
\newcolumntype{Y}{>{\raggedright\arraybackslash}X}
\usepackage[leftcaption]{sidecap}
%%%%% NEW MATH DEFINITIONS %%%%%

\usepackage{amsmath,amsfonts,bm}

% Mark sections of captions for referring to divisions of figures
\newcommand{\figleft}{{\em (Left)}}
\newcommand{\figcenter}{{\em (Center)}}
\newcommand{\figright}{{\em (Right)}}
\newcommand{\figtop}{{\em (Top)}}
\newcommand{\figbottom}{{\em (Bottom)}}
\newcommand{\captiona}{{\em (a)}}
\newcommand{\captionb}{{\em (b)}}
\newcommand{\captionc}{{\em (c)}}
\newcommand{\captiond}{{\em (d)}}

% Highlight a newly defined term
\newcommand{\newterm}[1]{{\bf #1}}


% Figure reference, lower-case.
\def\figref#1{figure~\ref{#1}}
% Figure reference, capital. For start of sentence
\def\Figref#1{Figure~\ref{#1}}
\def\twofigref#1#2{figures \ref{#1} and \ref{#2}}
\def\quadfigref#1#2#3#4{figures \ref{#1}, \ref{#2}, \ref{#3} and \ref{#4}}
% Section reference, lower-case.
\def\secref#1{section~\ref{#1}}
% Section reference, capital.
\def\Secref#1{Section~\ref{#1}}
% Reference to two sections.
\def\twosecrefs#1#2{sections \ref{#1} and \ref{#2}}
% Reference to three sections.
\def\secrefs#1#2#3{sections \ref{#1}, \ref{#2} and \ref{#3}}
% Reference to an equation, lower-case.
\def\eqref#1{equation~\ref{#1}}
% Reference to an equation, upper case
\def\Eqref#1{Equation~\ref{#1}}
% A raw reference to an equation---avoid using if possible
\def\plaineqref#1{\ref{#1}}
% Reference to a chapter, lower-case.
\def\chapref#1{chapter~\ref{#1}}
% Reference to an equation, upper case.
\def\Chapref#1{Chapter~\ref{#1}}
% Reference to a range of chapters
\def\rangechapref#1#2{chapters\ref{#1}--\ref{#2}}
% Reference to an algorithm, lower-case.
\def\algref#1{algorithm~\ref{#1}}
% Reference to an algorithm, upper case.
\def\Algref#1{Algorithm~\ref{#1}}
\def\twoalgref#1#2{algorithms \ref{#1} and \ref{#2}}
\def\Twoalgref#1#2{Algorithms \ref{#1} and \ref{#2}}
% Reference to a part, lower case
\def\partref#1{part~\ref{#1}}
% Reference to a part, upper case
\def\Partref#1{Part~\ref{#1}}
\def\twopartref#1#2{parts \ref{#1} and \ref{#2}}

\def\ceil#1{\lceil #1 \rceil}
\def\floor#1{\lfloor #1 \rfloor}
\def\1{\bm{1}}
\newcommand{\train}{\mathcal{D}}
\newcommand{\valid}{\mathcal{D_{\mathrm{valid}}}}
\newcommand{\test}{\mathcal{D_{\mathrm{test}}}}

\def\eps{{\epsilon}}


% Random variables
\def\reta{{\textnormal{$\eta$}}}
\def\ra{{\textnormal{a}}}
\def\rb{{\textnormal{b}}}
\def\rc{{\textnormal{c}}}
\def\rd{{\textnormal{d}}}
\def\re{{\textnormal{e}}}
\def\rf{{\textnormal{f}}}
\def\rg{{\textnormal{g}}}
\def\rh{{\textnormal{h}}}
\def\ri{{\textnormal{i}}}
\def\rj{{\textnormal{j}}}
\def\rk{{\textnormal{k}}}
\def\rl{{\textnormal{l}}}
% rm is already a command, just don't name any random variables m
\def\rn{{\textnormal{n}}}
\def\ro{{\textnormal{o}}}
\def\rp{{\textnormal{p}}}
\def\rq{{\textnormal{q}}}
\def\rr{{\textnormal{r}}}
\def\rs{{\textnormal{s}}}
\def\rt{{\textnormal{t}}}
\def\ru{{\textnormal{u}}}
\def\rv{{\textnormal{v}}}
\def\rw{{\textnormal{w}}}
\def\rx{{\textnormal{x}}}
\def\ry{{\textnormal{y}}}
\def\rz{{\textnormal{z}}}

% Random vectors
\def\rvepsilon{{\mathbf{\epsilon}}}
\def\rvtheta{{\mathbf{\theta}}}
\def\rva{{\mathbf{a}}}
\def\rvb{{\mathbf{b}}}
\def\rvc{{\mathbf{c}}}
\def\rvd{{\mathbf{d}}}
\def\rve{{\mathbf{e}}}
\def\rvf{{\mathbf{f}}}
\def\rvg{{\mathbf{g}}}
\def\rvh{{\mathbf{h}}}
\def\rvu{{\mathbf{i}}}
\def\rvj{{\mathbf{j}}}
\def\rvk{{\mathbf{k}}}
\def\rvl{{\mathbf{l}}}
\def\rvm{{\mathbf{m}}}
\def\rvn{{\mathbf{n}}}
\def\rvo{{\mathbf{o}}}
\def\rvp{{\mathbf{p}}}
\def\rvq{{\mathbf{q}}}
\def\rvr{{\mathbf{r}}}
\def\rvs{{\mathbf{s}}}
\def\rvt{{\mathbf{t}}}
\def\rvu{{\mathbf{u}}}
\def\rvv{{\mathbf{v}}}
\def\rvw{{\mathbf{w}}}
\def\rvx{{\mathbf{x}}}
\def\rvy{{\mathbf{y}}}
\def\rvz{{\mathbf{z}}}

% Elements of random vectors
\def\erva{{\textnormal{a}}}
\def\ervb{{\textnormal{b}}}
\def\ervc{{\textnormal{c}}}
\def\ervd{{\textnormal{d}}}
\def\erve{{\textnormal{e}}}
\def\ervf{{\textnormal{f}}}
\def\ervg{{\textnormal{g}}}
\def\ervh{{\textnormal{h}}}
\def\ervi{{\textnormal{i}}}
\def\ervj{{\textnormal{j}}}
\def\ervk{{\textnormal{k}}}
\def\ervl{{\textnormal{l}}}
\def\ervm{{\textnormal{m}}}
\def\ervn{{\textnormal{n}}}
\def\ervo{{\textnormal{o}}}
\def\ervp{{\textnormal{p}}}
\def\ervq{{\textnormal{q}}}
\def\ervr{{\textnormal{r}}}
\def\ervs{{\textnormal{s}}}
\def\ervt{{\textnormal{t}}}
\def\ervu{{\textnormal{u}}}
\def\ervv{{\textnormal{v}}}
\def\ervw{{\textnormal{w}}}
\def\ervx{{\textnormal{x}}}
\def\ervy{{\textnormal{y}}}
\def\ervz{{\textnormal{z}}}

% Random matrices
\def\rmA{{\mathbf{A}}}
\def\rmB{{\mathbf{B}}}
\def\rmC{{\mathbf{C}}}
\def\rmD{{\mathbf{D}}}
\def\rmE{{\mathbf{E}}}
\def\rmF{{\mathbf{F}}}
\def\rmG{{\mathbf{G}}}
\def\rmH{{\mathbf{H}}}
\def\rmI{{\mathbf{I}}}
\def\rmJ{{\mathbf{J}}}
\def\rmK{{\mathbf{K}}}
\def\rmL{{\mathbf{L}}}
\def\rmM{{\mathbf{M}}}
\def\rmN{{\mathbf{N}}}
\def\rmO{{\mathbf{O}}}
\def\rmP{{\mathbf{P}}}
\def\rmQ{{\mathbf{Q}}}
\def\rmR{{\mathbf{R}}}
\def\rmS{{\mathbf{S}}}
\def\rmT{{\mathbf{T}}}
\def\rmU{{\mathbf{U}}}
\def\rmV{{\mathbf{V}}}
\def\rmW{{\mathbf{W}}}
\def\rmX{{\mathbf{X}}}
\def\rmY{{\mathbf{Y}}}
\def\rmZ{{\mathbf{Z}}}

% Elements of random matrices
\def\ermA{{\textnormal{A}}}
\def\ermB{{\textnormal{B}}}
\def\ermC{{\textnormal{C}}}
\def\ermD{{\textnormal{D}}}
\def\ermE{{\textnormal{E}}}
\def\ermF{{\textnormal{F}}}
\def\ermG{{\textnormal{G}}}
\def\ermH{{\textnormal{H}}}
\def\ermI{{\textnormal{I}}}
\def\ermJ{{\textnormal{J}}}
\def\ermK{{\textnormal{K}}}
\def\ermL{{\textnormal{L}}}
\def\ermM{{\textnormal{M}}}
\def\ermN{{\textnormal{N}}}
\def\ermO{{\textnormal{O}}}
\def\ermP{{\textnormal{P}}}
\def\ermQ{{\textnormal{Q}}}
\def\ermR{{\textnormal{R}}}
\def\ermS{{\textnormal{S}}}
\def\ermT{{\textnormal{T}}}
\def\ermU{{\textnormal{U}}}
\def\ermV{{\textnormal{V}}}
\def\ermW{{\textnormal{W}}}
\def\ermX{{\textnormal{X}}}
\def\ermY{{\textnormal{Y}}}
\def\ermZ{{\textnormal{Z}}}

% Vectors
\def\vzero{{\bm{0}}}
\def\vone{{\bm{1}}}
\def\vmu{{\bm{\mu}}}
\def\vtheta{{\bm{\theta}}}
\def\va{{\bm{a}}}
\def\vb{{\bm{b}}}
\def\vc{{\bm{c}}}
\def\vd{{\bm{d}}}
\def\ve{{\bm{e}}}
\def\vf{{\bm{f}}}
\def\vg{{\bm{g}}}
\def\vh{{\bm{h}}}
\def\vi{{\bm{i}}}
\def\vj{{\bm{j}}}
\def\vk{{\bm{k}}}
\def\vl{{\bm{l}}}
\def\vm{{\bm{m}}}
\def\vn{{\bm{n}}}
\def\vo{{\bm{o}}}
\def\vp{{\bm{p}}}
\def\vq{{\bm{q}}}
\def\vr{{\bm{r}}}
\def\vs{{\bm{s}}}
\def\vt{{\bm{t}}}
\def\vu{{\bm{u}}}
\def\vv{{\bm{v}}}
\def\vw{{\bm{w}}}
\def\vx{{\bm{x}}}
\def\vy{{\bm{y}}}
\def\vz{{\bm{z}}}

% Elements of vectors
\def\evalpha{{\alpha}}
\def\evbeta{{\beta}}
\def\evepsilon{{\epsilon}}
\def\evlambda{{\lambda}}
\def\evomega{{\omega}}
\def\evmu{{\mu}}
\def\evpsi{{\psi}}
\def\evsigma{{\sigma}}
\def\evtheta{{\theta}}
\def\eva{{a}}
\def\evb{{b}}
\def\evc{{c}}
\def\evd{{d}}
\def\eve{{e}}
\def\evf{{f}}
\def\evg{{g}}
\def\evh{{h}}
\def\evi{{i}}
\def\evj{{j}}
\def\evk{{k}}
\def\evl{{l}}
\def\evm{{m}}
\def\evn{{n}}
\def\evo{{o}}
\def\evp{{p}}
\def\evq{{q}}
\def\evr{{r}}
\def\evs{{s}}
\def\evt{{t}}
\def\evu{{u}}
\def\evv{{v}}
\def\evw{{w}}
\def\evx{{x}}
\def\evy{{y}}
\def\evz{{z}}

% Matrix
\def\mA{{\bm{A}}}
\def\mB{{\bm{B}}}
\def\mC{{\bm{C}}}
\def\mD{{\bm{D}}}
\def\mE{{\bm{E}}}
\def\mF{{\bm{F}}}
\def\mG{{\bm{G}}}
\def\mH{{\bm{H}}}
\def\mI{{\bm{I}}}
\def\mJ{{\bm{J}}}
\def\mK{{\bm{K}}}
\def\mL{{\bm{L}}}
\def\mM{{\bm{M}}}
\def\mN{{\bm{N}}}
\def\mO{{\bm{O}}}
\def\mP{{\bm{P}}}
\def\mQ{{\bm{Q}}}
\def\mR{{\bm{R}}}
\def\mS{{\bm{S}}}
\def\mT{{\bm{T}}}
\def\mU{{\bm{U}}}
\def\mV{{\bm{V}}}
\def\mW{{\bm{W}}}
\def\mX{{\bm{X}}}
\def\mY{{\bm{Y}}}
\def\mZ{{\bm{Z}}}
\def\mBeta{{\bm{\beta}}}
\def\mPhi{{\bm{\Phi}}}
\def\mLambda{{\bm{\Lambda}}}
\def\mSigma{{\bm{\Sigma}}}

% Tensor
\DeclareMathAlphabet{\mathsfit}{\encodingdefault}{\sfdefault}{m}{sl}
\SetMathAlphabet{\mathsfit}{bold}{\encodingdefault}{\sfdefault}{bx}{n}
\newcommand{\tens}[1]{\bm{\mathsfit{#1}}}
\def\tA{{\tens{A}}}
\def\tB{{\tens{B}}}
\def\tC{{\tens{C}}}
\def\tD{{\tens{D}}}
\def\tE{{\tens{E}}}
\def\tF{{\tens{F}}}
\def\tG{{\tens{G}}}
\def\tH{{\tens{H}}}
\def\tI{{\tens{I}}}
\def\tJ{{\tens{J}}}
\def\tK{{\tens{K}}}
\def\tL{{\tens{L}}}
\def\tM{{\tens{M}}}
\def\tN{{\tens{N}}}
\def\tO{{\tens{O}}}
\def\tP{{\tens{P}}}
\def\tQ{{\tens{Q}}}
\def\tR{{\tens{R}}}
\def\tS{{\tens{S}}}
\def\tT{{\tens{T}}}
\def\tU{{\tens{U}}}
\def\tV{{\tens{V}}}
\def\tW{{\tens{W}}}
\def\tX{{\tens{X}}}
\def\tY{{\tens{Y}}}
\def\tZ{{\tens{Z}}}


% Graph
\def\gA{{\mathcal{A}}}
\def\gB{{\mathcal{B}}}
\def\gC{{\mathcal{C}}}
\def\gD{{\mathcal{D}}}
\def\gE{{\mathcal{E}}}
\def\gF{{\mathcal{F}}}
\def\gG{{\mathcal{G}}}
\def\gH{{\mathcal{H}}}
\def\gI{{\mathcal{I}}}
\def\gJ{{\mathcal{J}}}
\def\gK{{\mathcal{K}}}
\def\gL{{\mathcal{L}}}
\def\gM{{\mathcal{M}}}
\def\gN{{\mathcal{N}}}
\def\gO{{\mathcal{O}}}
\def\gP{{\mathcal{P}}}
\def\gQ{{\mathcal{Q}}}
\def\gR{{\mathcal{R}}}
\def\gS{{\mathcal{S}}}
\def\gT{{\mathcal{T}}}
\def\gU{{\mathcal{U}}}
\def\gV{{\mathcal{V}}}
\def\gW{{\mathcal{W}}}
\def\gX{{\mathcal{X}}}
\def\gY{{\mathcal{Y}}}
\def\gZ{{\mathcal{Z}}}

% Sets
\def\sA{{\mathbb{A}}}
\def\sB{{\mathbb{B}}}
\def\sC{{\mathbb{C}}}
\def\sD{{\mathbb{D}}}
% Don't use a set called E, because this would be the same as our symbol
% for expectation.
\def\sF{{\mathbb{F}}}
\def\sG{{\mathbb{G}}}
\def\sH{{\mathbb{H}}}
\def\sI{{\mathbb{I}}}
\def\sJ{{\mathbb{J}}}
\def\sK{{\mathbb{K}}}
\def\sL{{\mathbb{L}}}
\def\sM{{\mathbb{M}}}
\def\sN{{\mathbb{N}}}
\def\sO{{\mathbb{O}}}
\def\sP{{\mathbb{P}}}
\def\sQ{{\mathbb{Q}}}
\def\sR{{\mathbb{R}}}
\def\sS{{\mathbb{S}}}
\def\sT{{\mathbb{T}}}
\def\sU{{\mathbb{U}}}
\def\sV{{\mathbb{V}}}
\def\sW{{\mathbb{W}}}
\def\sX{{\mathbb{X}}}
\def\sY{{\mathbb{Y}}}
\def\sZ{{\mathbb{Z}}}

% Entries of a matrix
\def\emLambda{{\Lambda}}
\def\emA{{A}}
\def\emB{{B}}
\def\emC{{C}}
\def\emD{{D}}
\def\emE{{E}}
\def\emF{{F}}
\def\emG{{G}}
\def\emH{{H}}
\def\emI{{I}}
\def\emJ{{J}}
\def\emK{{K}}
\def\emL{{L}}
\def\emM{{M}}
\def\emN{{N}}
\def\emO{{O}}
\def\emP{{P}}
\def\emQ{{Q}}
\def\emR{{R}}
\def\emS{{S}}
\def\emT{{T}}
\def\emU{{U}}
\def\emV{{V}}
\def\emW{{W}}
\def\emX{{X}}
\def\emY{{Y}}
\def\emZ{{Z}}
\def\emSigma{{\Sigma}}

% entries of a tensor
% Same font as tensor, without \bm wrapper
\newcommand{\etens}[1]{\mathsfit{#1}}
\def\etLambda{{\etens{\Lambda}}}
\def\etA{{\etens{A}}}
\def\etB{{\etens{B}}}
\def\etC{{\etens{C}}}
\def\etD{{\etens{D}}}
\def\etE{{\etens{E}}}
\def\etF{{\etens{F}}}
\def\etG{{\etens{G}}}
\def\etH{{\etens{H}}}
\def\etI{{\etens{I}}}
\def\etJ{{\etens{J}}}
\def\etK{{\etens{K}}}
\def\etL{{\etens{L}}}
\def\etM{{\etens{M}}}
\def\etN{{\etens{N}}}
\def\etO{{\etens{O}}}
\def\etP{{\etens{P}}}
\def\etQ{{\etens{Q}}}
\def\etR{{\etens{R}}}
\def\etS{{\etens{S}}}
\def\etT{{\etens{T}}}
\def\etU{{\etens{U}}}
\def\etV{{\etens{V}}}
\def\etW{{\etens{W}}}
\def\etX{{\etens{X}}}
\def\etY{{\etens{Y}}}
\def\etZ{{\etens{Z}}}

% The true underlying data generating distribution
\newcommand{\pdata}{p_{\rm{data}}}
% The empirical distribution defined by the training set
\newcommand{\ptrain}{\hat{p}_{\rm{data}}}
\newcommand{\Ptrain}{\hat{P}_{\rm{data}}}
% The model distribution
\newcommand{\pmodel}{p_{\rm{model}}}
\newcommand{\Pmodel}{P_{\rm{model}}}
\newcommand{\ptildemodel}{\tilde{p}_{\rm{model}}}
% Stochastic autoencoder distributions
\newcommand{\pencode}{p_{\rm{encoder}}}
\newcommand{\pdecode}{p_{\rm{decoder}}}
\newcommand{\precons}{p_{\rm{reconstruct}}}

\newcommand{\laplace}{\mathrm{Laplace}} % Laplace distribution

\newcommand{\E}{\mathbb{E}}
\newcommand{\Ls}{\mathcal{L}}
\newcommand{\R}{\mathbb{R}}
\newcommand{\emp}{\tilde{p}}
\newcommand{\lr}{\alpha}
\newcommand{\reg}{\lambda}
\newcommand{\rect}{\mathrm{rectifier}}
\newcommand{\softmax}{\mathrm{softmax}}
\newcommand{\sigmoid}{\sigma}
\newcommand{\softplus}{\zeta}
\newcommand{\KL}{D_{\mathrm{KL}}}
\newcommand{\Var}{\mathrm{Var}}
\newcommand{\standarderror}{\mathrm{SE}}
\newcommand{\Cov}{\mathrm{Cov}}
% Wolfram Mathworld says $L^2$ is for function spaces and $\ell^2$ is for vectors
% But then they seem to use $L^2$ for vectors throughout the site, and so does
% wikipedia.
\newcommand{\normlzero}{L^0}
\newcommand{\normlone}{L^1}
\newcommand{\normltwo}{L^2}
\newcommand{\normlp}{L^p}
\newcommand{\normmax}{L^\infty}

\newcommand{\parents}{Pa} % See usage in notation.tex. Chosen to match Daphne's book.

\DeclareMathOperator*{\argmax}{arg\,max}
\DeclareMathOperator*{\argmin}{arg\,min}

\DeclareMathOperator{\sign}{sign}
\DeclareMathOperator{\Tr}{Tr}
\let\ab\allowbreak

% Custom shit
\newcommand{\bcal}[1]{\bm{\mathcal{#1}}}


\newcommand{\deq}{\mathrel{\mathop:}=} % aligned define equals
\newcommand{\deqrev}{=\mathrel{\mathop:}} % aligned define equals
\newcommand{\ft}{\sigma_\textsc{t}}
\newcommand{\fs}{\sigma}
\newcommand{\bfx}{\mathbf{x}}
\newcommand{\bfy}{\mathbf{y}}
\newcommand{\bft}{\mathbf{\theta}}
\newcommand{\bfe}{\boldsymbol{\varepsilon}}
\DeclareMathOperator{\NN}{N}
\DeclareMathOperator{\diag}{diag}
\newcommand{\bias}{B}
\newcommand{\Etrain}{E_\text{train}}
\newcommand{\Etest}{E_\text{test}}
\DeclareMathOperator{\tr}{tr}
\DeclareMathOperator{\erf}{erf}
\DeclareMathOperator{\Erf}{Erf}
\DeclareMathOperator{\relu}{ReLU}




% shortcut for inline equations and alignments
\newcommand{\eq}[1]{\begin{equation}#1\end{equation}}
\newcommand{\eqs}[1]{\begin{equation*}#1\end{equation*}}
\newcommand{\al}[1]{\begin{align}#1\end{align}}
\newcommand{\als}[1]{\begin{align*}#1\end{align*}}
\newcommand{\ca}[1]{\begin{cases}#1\end{cases}}

% ( round parentheses )
\newcommand{\p}[1]{({#1})}
\newcommand{\pb}[1]{\bigl({#1}\bigr)}
\newcommand{\pB}[1]{\Bigl({#1}\Bigr)}
\newcommand{\pbb}[1]{\biggl({#1}\biggr)}
\newcommand{\pBB}[1]{\Biggl({#1}\Biggr)}
\newcommand{\pa}[1]{\left({#1}\right)}

% [ square brackets ]
\newcommand{\q}[1]{[{#1}]}
\newcommand{\qb}[1]{\bigl[{#1}\bigr]}
\newcommand{\qB}[1]{\Bigl[{#1}\Bigr]}
\newcommand{\qbb}[1]{\biggl[{#1}\biggr]}
\newcommand{\qBB}[1]{\Biggl[{#1}\Biggr]}
\newcommand{\qa}[1]{\left[{#1}\right]}

% { curly braces }
\newcommand{\h}[1]{\{{#1}\}}
\newcommand{\hb}[1]{\bigl\{{#1}\bigr\}}
\newcommand{\hB}[1]{\Bigl\{{#1}\Bigr\}}
\newcommand{\hbb}[1]{\biggl\{{#1}\biggr\}}
\newcommand{\hBB}[1]{\Biggl\{{#1}\Biggr\}}
\newcommand{\ha}[1]{\left\{{#1}\right\}}

% | absolute value |
\newcommand{\abs}[1]{\lvert #1 \rvert}
\newcommand{\absb}[1]{\bigl\lvert #1 \bigr\rvert}
\newcommand{\absB}[1]{\Bigl\lvert #1 \Bigr\rvert}
\newcommand{\absbb}[1]{\biggl\lvert #1 \biggr\rvert}
\newcommand{\absBB}[1]{\Biggl\lvert #1 \Biggr\rvert}
\newcommand{\absa}[1]{\left\lvert #1 \right\rvert}

% || norm ||
\newcommand{\norm}[1]{\lVert #1 \rVert}
\newcommand{\normb}[1]{\bigl\lVert #1 \bigr\rVert}
\newcommand{\normB}[1]{\Bigl\lVert #1 \Bigr\rVert}
\newcommand{\normbb}[1]{\biggl\lVert #1 \biggr\rVert}
\newcommand{\normBB}[1]{\Biggl\lVert #1 \Biggr\rVert}
\newcommand{\norma}[1]{\left\lVert #1 \right\rVert}

% End Macros

\newcommand{\bfi}{\mathbf{i}}
\newcommand{\bfj}{\mathbf{j}}
\newcommand{\bfee}{\mathbf{e}}
\newcommand{\bfX}{\mathbf{X}}

\renewcommand{\cal}{\mathcal} % calligraphic
\newcommand{\e}{{\varepsilon}}
\renewcommand{\a}{\alpha}
\newcommand{\be}{\beta}

\newcommand{\V}{\mathbb{V}}

\newcommand{\x}{\times}
\newcommand{\cD}{\mathcal{D}}
\newcommand{\cX}{\mathcal{X}}
\newcommand{\cY}{\mathcal{Y}}
\newcommand{\cL}{\mathcal{L}}
\newcommand{\cA}{\mathcal{A}}
\newcommand{\cT}{\mathcal{T}}




\usepackage{caption}
%\captionsetup{type=figure}

\graphicspath{{media/}} 
% Definitions of handy macros can go here

\newcommand{\dataset}{{\cal D}}
\newcommand{\fracpartial}[2]{\frac{\partial #1}{\partial  #2}}

% Heading arguments are {volume}{year}{pages}{date submitted}{date published}{paper id}{author-full-names}

\usepackage{lastpage}
%\jmlrheading{23}{2022}{1-\pageref{LastPage}}{1/21; Revised 5/22}{9/22}{21-0000}{Author One and Author Two}

% Short headings should be running head and authors last names

%\ShortHeadings{Sample JMLR Paper}{One and Two}
%\firstpageno{1}
\newcommand{\imsize}{0.22\textwidth}

%%%%%%%%%%%%%%%%%%%%%%%%%%%%%%%%%%%%%%%%%%%%%%%%%%%%%%%%%
%% -- TIKZ DEFINITION -- %%
\pgfdeclarelayer{background}
\pgfdeclarelayer{foreground}
\pgfsetlayers{background,main,foreground}

% Define block styles used later

\tikzstyle{sensor}=[draw, fill=blue!20, text width=5em, 
    text centered, minimum height=2.5em,drop shadow]
\tikzstyle{ann} = [above, text width=5em, text centered]
\tikzstyle{wa} = [sensor, text width=10em, fill=red!20, 
    minimum height=6em, rounded corners, drop shadow]
\tikzstyle{sc} = [sensor, text width=13em, fill=red!20, 
    minimum height=10em, rounded corners, drop shadow]

% Define distances for bordering
\def\blockdist{2.3}
\def\edgedist{2.5}
\definecolor{bg}{gray}{0.95}

\usetikzlibrary{decorations.pathmorphing}
\usetikzlibrary{3d,perspective}
%%%%%%%%%%%%%%%%%%%%%%%%%%%%%%%%%%%%%%%%%%%%%%%%%%%%%%%%%

\begin{document}

% first row: images 1–4
%\begin{figure}
%\centering
%
%% first row
%%\subfloat[][$n=12$]{\includegraphics[width=\imsize]{img/%%descent_devel_t1.png}\label{fig:1a2}}\enspace
%%    \subfloat[][$n=24$]{\includegraphics[width=\imsize]{img/%%descent_devel_t2.png}\label{fig:1b2}}\enspace
%%    \subfloat[][$n=36$]{\includegraphics[width=\imsize]{img/%%descent_devel_t3.png}\label{fig:1c2}}\enspace
%%    \subfloat[][$n=50$]{\includegraphics[width=\imsize]{img/%%descent_devel_t4.png}\label{fig:1d2}} \\
%%
%%    % second row
%%    \subfloat[][$n=76$]{\includegraphics[width=\imsize]{img/%%descent_devel_t5.png}\label{fig:2a2}}\enspace
%%    \subfloat[][$n=82$]{\includegraphics[width=\imsize]{img/%%descent_devel_t6.png}\label{fig:2b2}}\enspace
%%    \subfloat[][$n=90$]{\includegraphics[width=\imsize]{img/%%descent_devel_t7.png}\label{fig:2c2}}\enspace
%%    \subfloat[][$n=100$]{\includegraphics[width=\imsize]{img/%%descent_devel_t8.png}\label{fig:2d2}} \\
%%
%%    % third row
%%    \subfloat[][$n=106$]{\includegraphics[width=\imsize]{img/%%descent_devel_t9.png}\label{fig:3a2}}\enspace
%%    \subfloat[][$n=112$]{\includegraphics[width=\imsize]{img/%%descent_devel_t10.png}\label{fig:3b2}}\enspace
%%    \subfloat[][$n=121$]{\includegraphics[width=\imsize]{img/%%descent_devel_t11.png}\label{fig:3c2}}
%
%% first row
%\subfloat[][$n=12$]{\includegraphics[width=\imsize]{img2/%risk_curve_n12.png}\label{fig:2a}}\enspace
%\subfloat[][$n=24$]{\includegraphics[width=\imsize]{img2/%risk_curve_n24.png}\label{fig:2b}}\enspace
%\subfloat[][$n=36$]{\includegraphics[width=\imsize]{img2/%risk_curve_n36.png}\label{fig:2c}}\enspace
%\subfloat[][$n=50$]{\includegraphics[width=\imsize]{img2/%risk_curve_n50.png}\label{fig:2d}} \\
%
%% second row
%\subfloat[][$n=76$]{\includegraphics[width=\imsize]{img2/%risk_curve_n76.png}\label{fig:2e}}\enspace
%\subfloat[][$n=82$]{\includegraphics[width=\imsize]{img2/%risk_curve_n82.png}\label{fig:2f}}\enspace
%\subfloat[][$n=90$]{\includegraphics[width=\imsize]{img2/%risk_curve_n90.png}\label{fig:2g}}\enspace
%\subfloat[][$n=100$]{\includegraphics[width=\imsize]{img2/%risk_curve_n100.png}\label{fig:2h}} \\
%
%% third row
%\subfloat[][$n=106$]{\includegraphics[width=\imsize]{img2/%risk_curve_n106.png}\label{fig:2i}}\enspace
%\subfloat[][$n=112$]{\includegraphics[width=\imsize]{img2/%risk_curve_n112.png}\label{fig:2j}}\enspace
%\subfloat[][$n=121$]{\includegraphics[width=\imsize]{img2/%risk_curve_n121.png}\label{fig:2k}}\enspace
%\subfloat[][$n=150$]{\includegraphics[width=\imsize]{img2/%risk_curve_n150.png}\label{fig:2l}} \\
%
%% fourth row
%\subfloat[][$n=231$]{\includegraphics[width=\imsize]{img2/%risk_curve_n231.png}\label{fig:2m}}\enspace
%\subfloat[][$n=250$]{\includegraphics[width=\imsize]{img2/%risk_curve_n250.png}\label{fig:2n}}\enspace
%\subfloat[][$n=256$]{\includegraphics[width=\imsize]{img2/%risk_curve_n256.png}\label{fig:2o}}\enspace
%\subfloat[][$n=280$]{\includegraphics[width=\imsize]{img2/%risk_curve_n280.png}\label{fig:2p}}
%
%
%
%\end{figure}

\begin{figure}[h!]
    \centering
    \begin{tikzpicture}[3d view={20}{15}]
        \foreach \Z in {0,...,4}
        {
            \begin{scope}[canvas is xy plane at z=-(\Z+ (0.5)*\Z),transform shape]
                \ifnum\Z>0 
                    \draw[semithick,gray,dashed] (0,0) coordinate(p\Z) rectangle (4,4);
                \fi 
                % Insert another TikZ picture dynamically for each layer
                \ifnum\Z=0 % Example: Add a specific illustration at layer 2
                \begin{scope}[scale=0.5]
                    \node (wa) [wa] at (5,1) {Core system};

                    \path (wa.west)+(-2.2,1.5) node (asr1) [sensor] {Sensor 1};
                    \path (wa.west)+(-2.2,0.5) node (asr2)[sensor] {Sensor 2};
                    \path (wa.west)+(-2.2,-1.0) node (dots)[ann] {$\vdots$}; 
                    \path (wa.west)+(-2.2,-2.0) node (asr3)[sensor] {Sensor $n$};    
                    \path (wa.south)+(0,-1.0) node (wo) [sensor] {Core components}; 
                   
                    \path (wa.east)+(\blockdist,0) node (vote) [sensor] {Actors $\mathcal{A}$};
                
                    \path [draw, ->] (wa.south) -- node [above] {} 
                        (wo.north) ;
                    
                    \path [draw, ->] (asr1.east) -- node [above] {} 
                        (wa.160) ;
                    \path [draw, ->] (asr2.east) -- node [above] {} 
                        (wa.180);
                    \path [draw, ->] (asr3.east) -- node [above] {} 
                        (wa.200);
                    \path [draw, ->] (wa.east) -- node [above] {} 
                        (vote.west);
                    \path (wa.south) +(0,-1.5) node (asrs) {};
                \end{scope}
                \fi
                \ifnum\Z=4
                \begin{scope}[scale=0.5]
                    \node (wa1) [sensor] at (4,4.2) {$M1$};
                    \path (wa1.west)+(-2.0,0) node (asr1) [sensor] {$M2$};
                    \path (wa1.east)+(2.0,0) node (asr2)[sensor] {$M3$};
                    \path (wa1.north)+(0,2.0) node (asr3)[sensor] {$M4$};   
                    \path (wa1.south)+(0,-2.0) node (asr4) [sensor] {$M5$}; 
                    \path [draw, ->] (wa1.west) -- node [above] {} 
                    (asr1.east) ;
                    \path [draw, ->] (wa1.east) -- node [above] {} 
                    (asr2.west) ;
                    \path [draw, ->] (wa1.north) -- node [above] {} 
                    (asr3.south) ;
                    \path [draw, ->] (wa1.south) -- node [above] {} 
                    (asr4.north) ;
            
                    \path [draw, ->] (asr1.north) -- node [above] {} 
                    (asr3.west) ;
                    \path [draw, ->] (asr3.east) -- node [above] {} 
                    (asr2.north) ;
                    \path [draw, ->] (asr2.south) -- node [above] {} 
                    (asr4.east) ;
                    \path [draw, ->] (asr4.west) -- node [above] {} 
                    (asr1.south) ;
                \end{scope}
                \fi
                \ifnum\Z=3
                \begin{scope}[scale=0.5]
                    \node (wa1) [sensor] at (4,4.2) {$M1$};
                    \path (wa1.west)+(-2.0,0) node (asr1) [sensor] {$M2$};
                    \path (wa1.east)+(2.0,0) node (asr2)[sensor] {$M3$};
                    \path (wa1.north)+(0,2.0) node (asr3)[sensor] {$M4$};   
                    \path (wa1.south)+(2.0,-2.0) node (asr4) [sensor] {$M5$}; 
                    \path (wa1.south)+(-2.0,-2.0) node (asr5) [sensor] {$M5$}; 
                    \path [draw, ->] (wa1.west) -- node [above] {} 
                    (asr1.east) ;
                    \path [draw, ->] (wa1.east) -- node [above] {} 
                    (asr2.west) ;
                    \path [draw, ->] (wa1.north) -- node [above] {} 
                    (asr3.south) ;
                    \path [draw, ->] (wa1.south) -- node [above] {} 
                    (asr4.north) ;
                    \path [draw, ->] (wa1.south) -- node [above] {} 
                    (asr5.north) ;
            
                    \path [draw, ->] (asr1.north) -- node [above] {} 
                    (asr3.west) ;
                    \path [draw, ->] (asr3.east) -- node [above] {} 
                    (asr2.north) ;
                    \path [draw, ->] (asr2.south) -- node [above] {} 
                    (asr4.east) ;
                    \path [draw, ->] (asr4.west) -- node [above] {} 
                    (asr5.east) ;
                    \path [draw, ->] (asr5.west) -- node [above] {} 
                    (asr1.south) ;
                \end{scope}
                \fi
                \ifnum\Z=2
                \begin{scope}[scale=0.5]
                    \node (wa1) [sensor] at (4,4.2) {$M1$};
                    \path (wa1.west)+(-2.0,2) node (asr1) [sensor] {$M2$};
                    \path (wa1.east)+(2.0,2) node (asr2)[sensor] {$M3$};
                    \path (wa1.south)+(0,-2.0) node (asr3)[sensor] {$M4$};   
            
                    \path [draw, ->] (wa1.west) -- node [above] {} 
                    (asr1.east) ;
                    \path [draw, ->] (wa1.east) -- node [above] {} 
                    (asr2.west) ;
                    \path [draw, ->] (wa1.south) -- node [above] {} 
                    (asr3.north) ;
            
            
                    \path [draw, ->] (asr1.south) -- node [above] {} 
                    (asr3.west) ;
                    \path [draw, ->] (asr3.east) -- node [above] {} 
                    (asr2.south) ;
                    \path [draw, ->] (asr2.west) -- node [above] {} 
                    (asr1.east) ;
                \end{scope}
                \fi
                \ifnum\Z=1
                \begin{scope}[scale=0.5]
                    \node (wa1) [sensor] at (4,4.2) {$M1$};
                    \path (wa1.west)+(-2.0,2) node (p2) [sensor] {$M2$};
                    \path (wa1.east)+(2.0,2) node (p3)[sensor] {$M3$};
                    \path (wa1.south)+(2,-2.0) node (p4)[sensor] {$M4$};  
                    \path (wa1.south)+(-2,-2.0) node (p5)[sensor] {$M5$};  
            
                    \path [draw, ->] (wa1.west) -- node [above] {} 
                    (p2.south) ;
                    \path [draw, ->] (wa1.east) -- node [above] {} 
                    (p3.south) ;
                    \path [draw, ->] (wa1.south) -- node [above] {} 
                    (p4.north) ;
                    \path [draw, ->] (wa1.south) -- node [above] {} 
                    (p5.north) ;
            
                    \path [draw, ->] (p2.east) -- node [above] {} 
                    (p3.west) ;
                    \path [draw, ->] (p3.south) -- node [above] {} 
                    (p4.north) ;
                    \path [draw, ->] (p4.west) -- node [above] {} 
                    (p5.east) ;
                    \path [draw, ->] (p5.north) -- node [above] {} 
                    (p2.south) ;
                \end{scope}
                \fi
            \end{scope}
            % Label for each layer
            \ifnum\Z>0
                \path (p\Z) node[left] {$\mathcal{L}_{\Z}([E_t])$};
            \fi
        }
    \end{tikzpicture}
\end{figure}


\end{document}
